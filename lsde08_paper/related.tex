\section{Related Work}
\label{sec:related}
During the past 9 years, from the creation of the Bitcoin Blockchain, there is a variety of analysis from many different aspects. Ron et al.~\cite{Ron}, have made a Quantitative Analysis of the full Bitcoin Blockchain transactions graph. However, their work differentiates from ours in terms of analysis. They are mainly focusing on exchange issues among users and they also deepen in the relation between transactions. On the other hand, McGinn et al.~\cite{McGinn} focus on activity of the transactions between users by providing a variety of interesting visualizations. They observe high frequency transaction patterns but also an uptrend of the Denial-of-Service-Attacks on the Bitcoin network. Even though their work is not relevant with ours in terms of graph-computations, they provide us very useful information about the transactions of Bitcoin.

The work from Bakayov and Custura~\cite{Bakayof} is in the same vein with ours but we approach the problem differently and compute different metrics. Moreover, they take into account the graph as a whole and they do not compute each year individually, which is a common-sense approach. However, most of their computations are not so heavy and except from Harmonic Centrality, they work with officialy optimized implementations maintained by the contributors of GraphX. 

Despite the fact that many researches are not about the Bitcoin Blockchain graph, they are also focusing on graph analysis and statistics. Quick et al.~\cite{Quick} compute also the Diameter and the Clustering Coefficient among other metrics. However, their work focuses on Social Network Analysis using Pregel-like large scale graph processing frameworks. They present several undirected graph algorithms for Social Network analysis and furthermore discuss various graph componentisation methods.

Finally, another approach is the one by Prat-Perez et al.~\cite{PratPerez} who compute metrics similar to ours but they are based on community-like analysis. Following their way we could split the graph of each year into communities by using the \textit{Label Propagation}, divide the communities in categories by size, discard communities with real small size and take a weighted sample from each size category. Such an approach would lead to a distribution of each metric very similar to the results if we have made the computation for the whole graph.
