This is how you refer to a paper!~\cite{quasar}. Check styles: {'TASK\_STAGING'}\textit{TaskStatus}, \texttt{--resources=\lq{gpgpus:\{gpu1, gpu2\}}\rq} 


This is how you write code!
{\tt \small
\begin{verbatim}
1. if(kernel_is_not_profiled){
2. //send kernel for profiling
3.     Profile(kernel);
4. }else{
5.     //use profiler's configurations
6.     set<profiles[kernel].blocks, profiles[kernel].threads>
7. }
8. blocks = initial_application_blocks;
9. threads = initial_application_threads;
10. blocks_start = initial_application_blocks;
11. 
12. Profile(kernel){
13.     while(true){
14.     
15.         init_exec_time = kernel.exec_time;
16.         blocks = blocks / 2;
17.         threads = threads /2;
18.         create_new_profiling_task = task;
19.         profile_task[task] = <blocks, threads>;
20.         issue(task);
21.         
22.         if(init_exec_time / task.exec_time > 0.9){
23.             blocks_end = blocks;
24.             best_combination = task;
25.         }
26.         
27.         if(blocks_end - blocks_start == 1){
28.             blocks = blocks_end;
29.             exit_flag = true;
30.         }
31.         
32.         if(exit_flag){
33.             profiles[task].blocks = best_combination.blocks;
34.             profiles[task].threads = best_combination.threads;
35.         }
36.     }
37. }                                                                                                                                                                                                        
\end{verbatim}
}



This is how you refer to an image! E.g: \fref{fig:performance_index} shows blah blah.
% This is how you insert and image
\begin{figure}[htb]
	\centering
	\includegraphics[width=0.8\linewidth]{./images/curves.eps}
	\caption{\perfindex{} shapes}
	\label{fig:performance_index}
\end{figure}



% List items
This is how you create a list of items:
\begin{enumerate}
\item Item 1
\item Item 2
\end{enumerate}



This is how you refer to a section. E.g: \secref{sec:intro} shows blah blah



Refer to table! \tref{tbl:profiler-overhead} 
\begin{table}[htb]
        \centering
        \caption{The overhead of the profiler for each application.}
        \label{tbl:profiler-overhead}
        \begin{tabular}{@{}ccc@{}}
                \toprule
                \toprule
                Blast & TPC-E & TPC-H\\
                \midrule
                500 secs & 250 secs & 450 secs\\
                \bottomrule
        \end{tabular}
\end{table}

\begin{table}[h!]
\centering
 \begin{tabular}{||c | c c||} 
 \hline
 Apps & Native & Profiler \\ [0.8ex] 
 \hline\hline
 MonteCarlo & 1.1581 & 1.2660 \\ 
 DarkGray & 0.0065 & 0.0064 \\
 Pathfinder & 0.2643 & 0.2655 \\
 Hotspot & 0.0211 & 0.0186 \\ [1ex]
 \hline
 \end{tabular}
\caption{Average execution time (in seconds) of kernels in native form and after profiling trasformations form.}
\label{table:1}
\end{table}


This is how you create one figure with 3 subfigures!
\begin{figure*}
\centering
\begin{subfigure}[b]{0.74\textwidth}
        \includegraphics[width=\textwidth]{./images/w1_zeus1.eps}
        \label{fig:w1_zeus1}
\end{subfigure}

\begin{subfigure}[b]{0.74\textwidth}
        \includegraphics[width=\textwidth]{./images/w1_zeus08.eps}
        \label{fig:w1_zeus08}
\end{subfigure}

\begin{subfigure}[b]{0.74\textwidth}
        \includegraphics[width=\textwidth]{./images/w1_simpleFW.eps}
        \label{fig:w1_simplefw}
\end{subfigure}
\caption{Percentages of allocated resources per server for the \textbf{first workload}. The first and the second plot refers to \name{} determining resources that their \perfindex{} is either equal to 1 or ranges between 0.8 and 0.85. The third plot corresponds to the case of a simple Mesos framework that users determine their resource requirements.}
\label{fig:first_workload}
\end{figure*}
