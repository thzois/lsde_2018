\section{Conclusions and Future Work}
\label{sec:conclusions}

The purpose of our work is to provide an insight of the evolution of the
Bitcoin Blockchain transactions graph. To achieve this, we analyze 98GBs of
data from which we keep only the necessary for the creation of the edges for each graph, managing 
to reduce them to 14GBs of parquet files. We split the whole graph
into subgraphs for each year and then for each month, so we can compute several
indicative for-the-graph metrics on certain time snapshots. Then, we create the
analogous visualizations which we upload at our website~\cite{website}. Our
results show that the graph tends to increase progressively each year, reaching
a peak in 2015. From the results it is obvious that since 2011 the activity of
the network was intense. However, we do not have a compelete image for 2017,
since our data are until March. 

Engaging with such a large graph-dataset makes the usage of Spark in
combination with GraphX decisive as they allow us to compute several metrics,
providing the necessary tools for graph processing. However, Bridge Ratio and
Diameter are too heavy computations and expensive algorithms. As the size of
the graph per year increases to billion edges, those tools we had in our
pleasure did not manage to provide any results, probably due to the lack of computation
power. 

As a future work we would like to approach the Bitcoin Blockchain graph in community analysis as described in the last paragraph of~\secref{sec:related}. This approach seems to apply better on large graphs and probably it would be faster to obtain the desired results.